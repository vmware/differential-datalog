\section{Introduction}\label{sec-introduction}

\subsection{Motivation}

Incremental algorithms tend to be significantly more complex than
their non-incremental versions.  In a nutshell, an incremental
algorithm propagates input changes to the output via all intermediate
steps.  This, in turn, requires (1) maintaining intermediate
computation results for each step, and (2) implementing an incremental
version of each operation, which, given an update to its input,
computes an update to its output.  Given that incremental computations
are ubiquitous throughout systems management software stacks, this
complexity affects the development cost, feature velocity, and,
performance of the system.

\emph{We argue that, instead of dealing with the complexity of
  incremental computation on a case-by-case basis, control plane
  developers should embrace programming tools that solve the problem
  once and for all.}  One such tool is \emph{Differential Datalog
  (DDlog)}~\cite{ddlog} -- a programming language that automates
incremental computation. A DDlog programmer only has to write an
algorithm for solving the original (non-incremental) problem.  Given
this description, the DDlog compiler generates an efficient
incremental implementation.

\subsection{Overview}

DDlog is a bottom-up, incremental, in-memory, typed Datalog engine for
writing application-integrated deductive database engines. DDlog is
based on Frank McSherry's excellent differential dataflow
library~\cite{differential-dataflow}.  DDlog is an open-source
project, hosted on github~\cite{ddlog}, with a permissive MIT-license.

\begin{description}
\item[Bottom-up]: DDlog starts from a set of ground facts (i.e., facts
  provided by the user) and computes all possible derived facts by
  following Datalog rules, in a bottom-up fashion. In contrast,
  top-down engines are optimized to answer individual user queries
  without computing all possible facts ahead of time. For example,
  given a Datalog program that computes pairs of connected vertices in
  a graph, a bottom-up engine maintains the set of all such pairs. A
  top-down engine, on the other hand, is triggered by a user query to
  determine whether a pair of vertices is connected and handles the
  query by searching for a derivation chain back to ground facts. The
  bottom-up approach is preferable in applications where all derived
  facts must be computed ahead of time and in applications where the
  cost of initial computation is amortized across a large number of
  queries.

\item[Incremental]: whenever the set of ground facts changes, DDlog
  only performs the minimum computation necessary to compute all
  changes in the derived facts. This has significant performance
  benefits for many queries.

\item[In-memory:] DDlog stores and processes data in memory. In a
  typical use case, a DDlog program is used in conjunction with a
  persistent database, with database records being fed to DDlog as
  ground facts and the derived facts computed by DDlog being written
  back to the database.

  At the moment, DDlog can only operate on databases that completely
  fit the memory of a single machine. (This may change in the future,
  as DDlog builds on the differential dataflow library that supports
  distributed computation over partitioned data).

\item[Typed]: Datalog in its classical textbook form it is more of a
  mathematical formalism than a practical tool for programmers. In
  particular, pure Datalog does not have concepts like data types,
  arithmetics, strings or functions.  To facilitate writing of safe,
  clear, and concise code, DDlog extends pure Datalog with:

  \begin{itemize}
    \item A powerful type system, including Booleans, unlimited
      precision integers, bitvectors, strings, tuples, and
      Haskell-style tagged unions (but with no recursive types).

    \item The ability to store sets and vectors as first-class values.
      
    \item Standard integer and bitvector arithmetic.

    \item A simple procedural language that allows expressing many
      computations natively in DDlog without resorting to external
      functions.

    \item String operations, including string concatenation and
      interpolation.
  \end{itemize}
  
  \item[Integrated]: while DDlog programs can be run interactively via
    a command line interface, its primary use case is to integrate
    with other applications that require deductive database
    functionality. A DDlog program is compiled into a Rust library
    that can be linked against a Rust, C/C++ or Java program (bindings
    for other languages can be easily added). This enables good
    performance, but somewhat limits the flexibility, as changes to
    the relational schema or rules require re-compilation.
\end{description}
