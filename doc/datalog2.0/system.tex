\section{DDlog Implementation}\label{sec-system}

\subsection{The DDlog compilation flow}

Figure~\ref{fig:compiler-flow} shows how DDlog programs are compiled
into native code.  The DDlog compiler generates Rust code, which is
linked with the Differential Dataflow
library~\cite{differential-dataflow} to produce a static or dynamic
library, which exposes an API to invoke the incremental computation at
runtime.  Differential Dataflow is the secret sauce behind DDlog's
excellent incremental performance.  It executes computations
represented as dataflow graphs, consisting of relational operators:
joins, antijoins, unions, aggregations, filters, and flatmaps, defined
over data collections.  Each operator has a highly optimized
incremental implementation, incorporating many generic optimizations.
Operators can be parameterized with arbitrary user-defined logic,
which enables a wide range of programs to be implemented on top of
Differential Dataflow.  Differential Dataflow is an independent
open-source project~\cite{differential-dataflow}, described in a
series of
publications~\cite{timely-dataflow,differential-dataflow-paper} and
online documents~\cite{dd-mdbook,dd-reference}.

\subsection{C and Java interfaces}

The compiled DDlog program exposes a transactional API to add and
delete records from input relations and notify the client about
changes in output relations at runtime.  The API is natively
implemented in Rust, with bindings available for other languages,
currently C and Java.  We use the latter to integrate DDlog in the NSC
control plane.  The Java API to DDlog is packaged as a JAR file, and
supplied together with a dynamically-linked library (DLL) that
comprises the native executable.  The Java API is program-independent,
but the native executable needs to be regenerated every time the DDlog
program is changed. We illustrate the Java API to DDlog with a
skeleton example in Figure~\ref{fig:javaapi}.

\begin{figure}
    \footnotesize
    \begin{lstlisting}[language=Java]
/* Java class that represents a record in the Dependency
 * relation. */
class Genealogy {
  String parent;
  String child;
}

/* Instantiate the DDlog program; add a record to the
 * Dependency relation. */
static void main() {
  DDlogAPI api = new DDlogAPI(2 /* threads */,
               r -> onCommit(r) /* callback */);
  /* Obtain handle to Dependency relation. */
  int table = api.getTable("Genealogy");
  /* Create an instance of Dependency class. */
  Genealogy g = new Genealogy();
  dep.parent = "Mike";
  dep.child  = "John";
  /* Create DDlog insert command. */
  DDlogCommand command = new DDlogCommand(
     DDlogCommand.Kind.Insert,
     table,
     dep);

  /* Execute command as an atomic transaction. */
  int exitcode = api.start();  // start transaction
  DDlogCommand[] commands = new DDlogCommands[1];
  commands[0] = command;
  exitcode = api.applyUpdates(commands);
  /* commit transaction; causes the callback to be 
   * invoked for each new or deleted output record */
  exitcode = api.commit();
  /* Terminate execution of the DDlog program. */
  api.stop();
}

static void onCommit(DDlogCommand command) {
  /* Callback invoked on commit once for each insertion or
   * deletion in an output relation. */
  int outputTable = command.tableId;
  if (command.kind == DDlogCommand.Kind.Insert) {
    Object value = command.getObject();
    // insert into outputTable
  } else {
    // delete from outputTable
  }
}
\end{lstlisting}
\caption{Example program showing the Java API to pass and receive data
  from a DDlog program.\label{fig:javaapi}.}
\end{figure}
