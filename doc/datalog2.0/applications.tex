\section{Applications}\label{sec:applications}

\subsection{Controller for Virtual Networks}

The most significant program DDlog program that we have written is a
reimplementation of the OVN controller, which is part of the virtual
network management software stack of Open virtual Switch
(OvS)~\cite{Pfaff-nsdi15}.

OVN translates a set of network management policies into OpenFlow
rules that have to be installed on the virtual switches in the
network.  The logic is very complicated, comprising tens of input,
output and intermediate relations.

The original program was written in C, and is not fully incremental.
The DDlog implementation has about 6000 lines of code, about the same
size as the original code-base, but it is fully incremental.
Translating this program proved to be quite challenging.  The original
program builds an in-memory object model of the database; in that
implementation joins with a primary key become just pointer
dereferences.  Moreover, some operations are implemented using
side-effects and are challenging to do in a declarative model; an
example is allocation of network addresses, where new addresses are
allocated using a counter, to prevent old addresses from being reused
for as long as possible.  The original program generates open-flow
rules encoded as strings, so this implementation makes extensive use
of the string interpolation facilities of DDlog.

Preliminary evaluations indicate that the incremental performance is
several orders of magnitude better than the original program for large
networks.

TODO: how well does it work.

\subsection{Program analysis}

We have evaluated DDlog on a large Datalog program written for the
Souffle Datalog compiler.  The Datalog program has performs many
compiler analyses simultaneously, computing their fixed-point.  We
have written a Python program that converts a large subset of the
Souffle Datalog syntax into DDlog (we have manually handled the
missing features); the input program has 580 lines.

The input data to the program consists of 3.5 million tuples.

TODO: how well does it work?

\subsection{Firewall management}

We have reimplemented a proprietary network management application in
DDlog.  The application manages a firewall in a network of switches
and virtual machines (VMs).  The firewall is driven by a centralized
policy.  The centralized policy is implemented in a distributed
fashion by the VMs and network switches, each of which performs
filtering using local rules.  When the centralized policy changes the
local rules have to be updated in all network devices.  The core
of this program is a graph reachability problem, which is written in a
few lines of DDlog.  The DDlog program outperformed in performance by
a factor of 3-8 a hand-written incremental Java program that had
thousands of lines of code.
