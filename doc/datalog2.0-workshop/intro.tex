\section{Introduction}\label{sec-introduction}

\paragraph{Motivation.}
Many real-world applications must update their output in response to input changes.  Consider, for
example, a cluster management system such as Kubernetes~\cite{kubernetes}, that configures cluster
nodes to execute a user-defined workload.  As the workload changes, e.g., container instances are
added or removed from the system, the configuration must change accordingly.  In a large cluster
computing configuration from scratch is prohibitively expensive.  Instead, modern cluster management
systems, including Kubernetes, apply changes incrementally, only updating state effected by the
change.

As another example, program analysis frameworks like Doop~\cite{Bravenboer-oopsla09} evaluate a set
of rules defined over the abstract syntax tree of the program.  Such an analyzer can be integrated
into an IDE to alert the developer as soon as a potential bug is introduced in the program.  This
requires re-evaluating the rules after every few keystrokes.  In order to achieve interactive
performance when working with very large code bases, the re-evaluation must occur incrementally,
preserving as much as possible intermediate results computed at earlier iterations.

Incremental algorithms tend to be significantly more complex than
their non-incremental versions.  An incremental algorithm must
propagate input changes to the output via all intermediate
computation steps.  This, in turn, requires (1) maintaining
intermediate computation results for each step, and (2) implementing
an incremental version of each operation, which, given an update to
its input, computes an update to its output.  Incremental computations
that operate on relational state are ubiquitous throughout systems
management software stacks.  The complexity of the incremental algorithms
greatly impacts the development cost, feature velocity,
maintainability, and performance of the control systems.

We argue that, instead of dealing with the complexity of incremental
computation on a case-by-case basis, developers should embrace
programming tools that solve the problem once and for all.  In this
paper we present one such tool --- Differential Datalog (DDlog)
--- a programming language that automates incremental computation.  A
DDlog programmer only has to write a Datalog specification for the
original (non-incremental) problem.  From this description the DDlog
compiler generates an efficient incremental implementation.

\paragraph{Overview.}
DDlog is a \emph{bottom-up}, \emph{incremental}, \emph{in-memory},
\emph{typed} Datalog engine for building \emph{embedded} deductive
databases.

\textbf{Bottom-up:} DDlog starts from a set of ground facts (provided
by the user) and computes all possible derived facts by following
Datalog rules, in a bottom-up fashion.  (In contrast, top-down engines
are optimized to answer individual user queries without computing all
possible facts ahead of time.)

\textbf{Incremental:} whenever presented with changes to the ground
facts, DDlog only performs the minimum computation necessary to
compute all changes in the derived facts.  This has significant
performance benefits, and only produces output of minimum size,
also reducing communication requirements.  DDlog evaluation is \emph{always
  incremental}; non-incremental (traditional) evaluation can be
implemented as a special case, starting from empty relations.

\textbf{In-memory:} DDlog stores and processes data in
memory\footnote{In a typical use case, a DDlog program is used in
  conjunction with a persistent database; database records are fed to
  DDlog as inputs and the derived facts computed by DDlog are written
  back to the database; DDlog does not include a storage engine.}.  At
the moment, DDlog keeps all the data in the memory of a single
machine\footnote{The core engine of DDlog is differential
  dataflow~\cite{differential-dataflow}, which supports distributed
  computation over partitioned data; we may add this capability in the
  future.}.

\textbf{Typed:} Pure Datalog does not have concepts like data types,
arithmetic, strings or functions.  To facilitate writing of safe,
maintainable, and concise code, DDlog extends Datalog with:
\begin{itemize}
\item A powerful type system, including Booleans, unlimited
  precision integers, bit-vectors, strings, tuples, and
  Haskell-style tagged unions.

\item Standard integer and bit-vector arithmetic.

\item A simple functional language containing functions that allows
  expressing many computations over these data-types in DDlog without
  resorting to external functions.

\item String operations, including string concatenation and
  interpolation.

\item The ability to store and manipulate sets, vectors, and maps as
  first-class values in relations, including performing aggregations.
\end{itemize}

\textbf{Embedded:} while DDlog programs can be run interactively via a
command line interface, the primary use case is to run DDlog in the
same address space with an application that requires deductive
database functionality.  A DDlog program is compiled into a Rust
library that can be linked against a Rust, C/C++ or Java program
(bindings for other languages can be easily added).

DDlog is an open-source project, hosted on github~\cite{ddlog} using
an MIT-license.
