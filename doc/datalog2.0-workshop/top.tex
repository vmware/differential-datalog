\documentclass{svproc}

\usepackage{float}
\usepackage[T1]{fontenc}
\usepackage{url}
\usepackage{graphicx}
\usepackage{wrapfig}
\usepackage{listings}
\usepackage{color}

\renewcommand{\floatpagefraction}{.95}
\renewcommand{\topfraction}{.95}

\definecolor{blue}{rgb}{0.13,0.13,1}
\definecolor{green}{rgb}{0,0.5,0}
\definecolor{red}{rgb}{0.9,0,0}
\definecolor{grey}{rgb}{0.46,0.45,0.48}

\lstset{language=Java,
  commentstyle=\color{green},
  keywordstyle=\color{blue},
  stringstyle=\color{red},
  basicstyle=\ttfamily}

\lstdefinelanguage{ddlog}{
  language=Java, % we base it on Java, just for comments
  morekeywords={input, output, typedef, relation, typedef, bool,
    string, bit, extern, function, var, for, match, skip, in,
    Aggregate, FlatMap},
  deletestring=[b]{'}
}

\newcommand{\LR}[1]{\textbf{\color{red}LR: #1}}

\author{
        Leonid Ryzhyk \and
        Mihai Budiu}
\institute{VMware Research}

\title{Differential Datalog}

\date{}

\begin{document}
\maketitle

\section{Abstract}

%In distributed systems, the control plane is responsible for
%reconfiguring the system in response to external events.  Due to
%performance and scalability requirements, it is impractical to
%recompute the entire configuration on every update.  Instead,
%developers employ \emph{incremental algorithms} that update their
%outputs in response to input changes.  Such algorithms are notoriously
%hard to develop, leading to complex and inflexible implementations.
%(This is the classic ``incremental view update''
%problem~\cite{Gupta-sigmod93}).


Many real-world applications of deductive databases require
incrementally updating output tables in response to changes to input
tables.
To address this need, we have developed Differential Datalog
(DDlog), a dialect of Datalog that automates incremental computation.
A DDlog programmer only has to write a specification for the
original (non-incremental) problem.  Given this description, the DDlog
compiler generates an efficient incremental implementation.  The DDlog
runtime automatically maintains indexes required to efficiently
compute output updates.

The DDlog language is targeted for system builders.  In consequence,
the language emphasizes usability, by providing a rich type system,
a powerful expression language, including string manipulation, arithmetics,
and control flow constructs.  Finally, DDlog offers an
alternative syntax for writing Datalog rules, which has a more
imperative flavor.

\section{Introduction}\label{sec-introduction}

\paragraph{Motivation.}
Many real-world applications must update their output in response to input changes.  Consider, for
example, a cluster management system such as Kubernetes~\cite{kubernetes} that configures cluster
nodes to execute a user-defined workload.  As the workload changes, e.g., container instances are
added or removed from the system, the configuration must change accordingly.  In a large cluster,
computing configuration from scratch is prohibitively expensive.  Instead, modern cluster management
systems, including Kubernetes, apply changes incrementally, only updating state effected by the
change.

As another example, program analysis frameworks like Doop~\cite{Bravenboer-oopsla09} evaluate a set
of rules defined over the abstract syntax tree of the program.  Such an analyzer can be integrated
into an IDE to alert the developer as soon as a potential bug is introduced in the program.  This
requires re-evaluating the rules after every few keystrokes.  In order to achieve interactive
performance when working with very large code bases, the re-evaluation must occur incrementally,
preserving as much as possible intermediate results computed at earlier iterations.

Incremental algorithms tend to be significantly more complex than
their non-incremental versions.  An incremental algorithm must
propagates input changes to the output via all intermediate
computation steps.  This, in turn, requires (1) maintaining
intermediate computation results for each step, and (2) implementing
an incremental version of each operation, which, given an update to
its input, computes an update to its output.  Incremental computations
that operate on relational state are ubiquitous throughout systems
management software stacks.  The complexity of the incremental algorithms
greatly impacts the development cost, feature velocity,
maintainability, and the performance of the control systems.

We argue that, instead of dealing with the complexity of incremental
computation on a case-by-case basis, developers should embrace
programming tools that solve the problem once and for all.  In this
paper we present one such tool --- \emph{Differential Datalog (DDlog)}
--- a programming language that automates incremental computation.  A
DDlog programmer only has to write a Datalog specification for the
original (non-incremental) problem.  Given this description, the DDlog
compiler generates an efficient incremental implementation.

\paragraph{Overview.}
DDlog is a \emph{bottom-up}, \emph{incremental}, \emph{in-memory}, \emph{typed} Datalog engine for
writing \emph{embedded} deductive databases.

\textbf{Bottom-up:} DDlog starts from a set of ground facts (provided
by the user) and computes all possible derived facts by following
Datalog rules, in a bottom-up fashion.  (In contrast, top-down engines
are optimized to answer individual user queries without computing all
possible facts ahead of time.)

\textbf{Incremental:} whenever presented with changes to the ground
facts, DDlog only performs the minimum computation necessary to
compute all changes in the derived facts.  This has significant
performance benefits, and only produces output of minimum size,
reducing communication requirements.

\textbf{In-memory:} DDlog stores and processes data in
memory\footnote{In a typical use case, a DDlog program is used in
  conjunction with a persistent database; database records are fed to
  DDlog as inputs and the derived facts computed by DDlog are written
  back to the database; DDlog does not include a storage engine.}.  At
the moment, DDlog keeps all the data in the memory of a single
machine\footnote{This may change, since the core engine of DDlog is
  differential dataflow~\cite{dd} supports distributed
  computation over partitioned data.}.

\textbf{Typed:} Pure Datalog does not have concepts like data types,
arithmetics, strings or functions.  To facilitate writing of safe,
clear, and concise code, DDlog extends Datalog with:
\begin{itemize}
\item A powerful type system, including Booleans, unlimited
  precision integers, bitvectors, strings, tuples, and
  Haskell-style tagged unions (but without recursive types).

\item The ability to store and manipulate sets, vectors, and maps as
  first-class values in relations, including performing aggregations.

\item Standard integer and bitvector arithmetic.

\item A simple functional language containing functions that allows
  expressing many computations over these datatypes in DDlog without
  resorting to external functions.

\item String operations, including string concatenation and
  interpolation.
\end{itemize}

\textbf{Embedded:} while DDlog programs can be run interactively via a
command line interface, the primary use case is to run DDlog in the
same address space with an application that requires deductive
database functionality.  A DDlog program is compiled into a Rust
library that can be linked against a Rust, C/C++ or Java program
(bindings for other languages can be easily added).

DDlog is an open-source project, hosted on github~\cite{ddlog} using
an MIT-license.

\section{Differential Datalog (DDlog)}\label{sec-ddlog}

A DDlog program operates on typed relations.  The programmer defines a
set of rules to compute a set of output relations based on input
relations (Figure~\ref{fig:differential}).  Rules are evaluated
incrementally: given a set of \emph{changes} to the input relations
(insertions or deletions), DDlog produces a set of changes to the
output relations (expressed also as insertions or deletions).

\begin{figure}[t]
    \center
    \includegraphics[width=0.5\columnwidth,clip=true,trim=0in 4.4in 6.5in 0in]{differential.pdf}
    \caption{Incremental evaluation of a Datalog program.\label{fig:differential}}
\end{figure}

In this section we give a brief overview of the language; we refer the
reader to the DDlog language reference~\cite{ddlog-manual} and the DDlog
tutorial~\cite{ddlog-tutorial} for a detailed presentation of language
features.

%\subsection{Syntax}
%
%DDlog is case-sensitive.  Relation, constructor, and type variable
%names must start with upper-case ASCII letters; variable, function,
%and argument names must start with lower-case ASCII letters or
%underscore.  A type variable name must be prefixed with a tick
%(\texttt{'}). A type name can start with either an upper-case or a
%lower-case letter or underscore.  Names must be unique within a scope.

\subsection{Type system}

DDlog is a strongly-typed language.  DDlog performs type inference,
but users can specify the types of expressions explicitly as well.
All programs are type-checked statically.  The type system is inspired
by Haskell, and supports a rich set of types.  Base types include
Booleans, bit-strings, e.g., \texttt{bit<32>}, infinite-precision
integers \texttt{bigint}, UTF-8 strings.  Derived types are tuples,
structures, and tagged unions (which generalize enumerated types).  We
do not allow defining recursive types like lists or trees; however
it provides three built-in collection types: maps, sets, and arrays
(Section~\ref{sec:collections}).

Figure~\ref{fig:types} shows several type declarations.  DDlog
supports also generic types; type variables are preceded by a tick:
\texttt{'A}.  The language contains a built-in reference type
\texttt{Ref<'T>}.  Unlike other languages, two references are equal if
the objects referred are equal; thus references do not alter the
nature of Datalog in any significant way.  References can be used to
reduce memory consumption when complex objects are stored in multiple
relations.

\begin{figure}[t]
  \small
\begin{lstlisting}[language=ddlog]
// A tagged-union type
typedef IPAddress = IPv4Address{ipv4addr: bit<32>}
                  | IPv6Address{ipv6addr: bit<128>}
// A generic option type
typedef Option<'A> = None
                   | Some{value: 'A}
typedef OptionalIPAddress = Option<IPAddress>                   
\end{lstlisting}
\caption{Some type declarations in DDlog.\label{fig:types}}
\end{figure}

\subsection{Relations}

Relations are strongly typed; the value in each column must have a
statically-determined type.  There are three kinds of relations in
DDlog:
\begin{description}
\item[Input relations:] the contents of these relations is provided by
  the environment of the program.
\item[Output relations:] these must be computed by the DDlog program, and the
  DDlog runtime will inform the environment of any changes that
  occur in these relations.
\item[Intermediate results:] these must be also computed by the DDlog
  program, but the environment cannot query the contents of these
  relations.
\end{description}

Figure~\ref{fig:relations} shows example relation declarations.  An
input relation may declare an optional primary key --- which is a set
of columns that can be used to delete entries more efficiently,
specifying only the key.

\begin{figure}[t]
  \small
  \begin{lstlisting}[language=ddlog]
input relation Edge(from: node_t, to: node_t)
               primary key (e) e.from
output relation Path(src: node_t, dst: node_t)
  \end{lstlisting}
  \caption{Example relation declarations.\label{fig:relations}}
\end{figure}

\subsection{Rules}

DDlog rules are composed of standard Datalog operators: joins, antijoins, and
unions, illustrated in Figure~\ref{fig:rules}, as
well as aggregation, and flatmap, discussed in Section~\ref{sec:collections}.
DDlog allows recursive rules with stratified negation: intuitively, a DDlog
relation cannot recursively depend on its own negation.

\begin{figure}[t]
  \small
  \begin{lstlisting}[language=ddlog]
/* The Path relation is computed as a union of two rules */

// Rule 1: base case
Path(x, y) :- Edge(x,y).

// Rule 2: recursive step: join Path relation with Edge
// followed by an antijoin with Exclude to ignore nodes
// in the Exclude relation.
Path(x, z) :- Path(x, w), Edge(w, z), not Exclude(z).
  \end{lstlisting}
  \caption{Example DDlog rules that compute the set of paths in a graph.\label{fig:rules}}
\end{figure}

\subsection{Expressions}

Much of DDlog's power stems from its ability to perform complex computation
inside rules.  For example, the rule in Figure~\ref{fig:area} computes
an inner join of height and width tables using the object id column and
then computes the area of the object as the product of its height and width.
This is in contrast to the textbook Datalog that is limited to selecting
columns from existing relations.

\begin{figure}[t]
  \small
  \begin{lstlisting}[language=ddlog]
input relation Height(object_id: int, height: int)
input relation Width(object_id: int, width: int)
output relation Area(object_id: int, area: int)

// Compute the area of an object as the product of
// its height and width.
Area(oid, area) :- Height(oid, h), Width(oid, w),
                   var area = w * h.
  \end{lstlisting}
  \caption{Using expressions in a rule.\label{fig:area}}
\end{figure}

The full DDlog expression language supports arithmetics, string manipulation, control
flow constructs and function calls.

%DDlog is an expression-oriented language; programs are composed of
%expressions, that evaluate to ground values.  Sequential composition
%of two expressions (written using semicolon) evaluates to the last
%expression.

\paragraph{Arithmetics}
The arithmetic types (\texttt{bigint} and \texttt{bit<N>}) provide the
standard arithmetic operations, as well as bit-wise operations, bit selection
\texttt{v[15:8]}, shifting, and concatenation.

\paragraph{Strings} All primitive types contain built-in conversions to strings, and users
can implement string conversion functions for user-defined types
(like Java's toString() method).

Expressions enclosed within \texttt{\$\{...\}} in a string literal are
\emph{interpolated}: they are evaluated at run-time, converted to
strings and substituted; this is a feature inspired by JavaScript; for
example \texttt{"x+y=\$\{x+y\}"}.

\paragraph{Control flow}

DDlog is an expression-oriented language, meaning that any expression
evaluates to a ground value and can be used, e.g., in the righ-hand side
of an assignment.  Similar to other expression-oriented languages,
DDlog supports \texttt{if-else} expressions, where the \texttt{else} branch
must be present.

Inspired by ML and Haskell, the DDlog \texttt{match} expression
simultaneously performs pattern-matching against type constructors or
values and value binding.  Figure~\ref{fig:function} shows a
\texttt{match} expression that uses a nested pattern to extract a byte from
\texttt{OptionalIPAddress}.  This pattern binds the \texttt{addr}
variable.

\begin{figure}[t]
  \small
  \begin{lstlisting}[language=ddlog]
function lastByte(a: OptionalIPAddress): bit<8> = {
  match (a) {
    None -> 0,
    Some{IPv4Address{.ipv4addr = addr}} -> addr[7:0],
    Some{IPv6Address{.ipv6addr = addr}} -> addr[7:0]
  }
}
  \end{lstlisting}
\caption{A DDlog function using pattern matching.\label{fig:function}}
\end{figure}

Pattern matching can also be used directly in the body of a rule.  For example,
the following rule extracts only IPv6 addresses from the \texttt{Host}
relation.

\begin{lstlisting}[language=ddlog]
IPv6Addr(addr):-Host(.address=Some{IPv6Address{addr}}).
\end{lstlisting}

Finally, programmers can use \texttt{for} loops to iterate
over elements of collections, as explained in Section~\ref{sec:collections}
below.

\paragraph{Local variables}

Local variables are used to store intermediate results of computations.
In DDlog, local variables can be introduced in three different contexts.  First,
variables can be defined directly in the body of a rule, e.g., the \texttt{area}
variable in Figure~\ref{fig:area}.  Second, they can be defined inside
an expression, e.g., the \texttt{res} variable in Figure~\ref{fig:collections}.
Third, a variable can be defined in a \texttt{match} pattern,
as in Figure~\ref{fig:function}.
A local variable is visible within the syntactic scope where it was defined.

\subsection{Functions}

DDlog functions encapsulate pure (side-effect-free) computations.  
An example function is shown in
Figure~\ref{fig:function}.  Recursive functions are not supported.
Users and libraries can declare prototypes of \texttt{extern}
functions, which must be implemented outside of DDlog (e.g., in Rust),
and linked against the DDlog program at link time.  The compiler
assumes that extern functions are pure.

\subsection{Collections}\label{sec:collections}

The DDlog standard library contains three built-in generic collection
types (implemented natively in Rust): \texttt{Vec<'T>},
\texttt{Set<'T>} and \texttt{Map<'T>}.  Values of these types can be
stored as first-class values within relations.  Equality for values
of these types is defined as expected, element-wise.  In theory such
types are not necessary, since collections within relations can be
represented using separate relations.  We have
introduced them into the language because many practical applications
have data models that contain nested collections; by supporting
collection-valued columns natively in DDlog we can more easily
interface with such applications, without the need to write glue code
to convert collections back and forth into separate relations using
foreign keys.

%The downside of using collections as relation values is
%that DDlog does not compute incremental changes of collections (only
%incremental changes to relations).

Figure~\ref{fig:collections} shows the declaration in DDlog of an
external function which splits a string into substrings using a
separator; this function returns a vector of strings.

\begin{figure}[t]
  \small
  \begin{lstlisting}[language=ddlog]
// declare external function returning a vector of strings
extern function split(s: string, sep: string): Vec<string>
// DDlog function to concatenate all elements of a vector
function concat(s: Vec<string>, sep: string): string = {
  var res = "";
  for (e in s) {
    res = if (res != "") res + sep else res;
    res = res + e
  };
  res   // last value is function evaluation result
}

input relation Phrases(p: string)
relation Words(w: string)
// Words contains all words that appear in some phrase
Words(w) :- Phrases(p), var w = FlatMap(split(p, " ")).

// Shortest path between each pair of points x, y
// (x, y) is the key for grouping
// min is the function used to aggregate data in each group
ShortestPath(x, y, min_cost) :- Path(x, y, cost),
                 var min_cost = Aggregate((x, y), min(cost)).
\end{lstlisting}
\caption{Operations on collections: iteration, flattening,
  aggregation.\label{fig:collections}}
\end{figure}

\texttt{for} loops can be used to iterate over elements in
collections.  Figure~\ref{fig:collections} shows an implementation of
the function \texttt{concat}, the inverse of \texttt{split} using a
loop.

The \texttt{FlatMap} operator can be used to flatten
a collection into a set of DDlog records, as illustrated in the definition of
relation \texttt{Words} in Figure~\ref{fig:collections}.

The \texttt{Aggregate} operator can be used to evaluate the equivalent
of SQL groupby-aggregate queries.  The aggregate operator has two
arguments: a key function, and an aggregation function.  The
aggregation function receives a group of
records that share the same key.  The
\texttt{ShortestPath} relation in Figure~\ref{fig:collections} is
computed using aggregation.

\subsection{Module system}

DDlog offers a simple module system, inspired by Haskell and Python,
which allows importing definitions (types, functions, relations) from
multiple files.  The user can add imported definitions directly into the
name space of the importing module or keep them in a separate name space to
prevent name clashes.  Similar to Java packages, module names are
hierarchical and the module name hierarchy must match the paths on the
the filesystem where modules are stored.  The directive \texttt{import
  library.module} will load the module from file
\texttt{library/module.dl}.  The standard library itself is a module
named \texttt{std}.

\subsection{``Imperative'' relation definitions}

We have also defined an alternative syntax for rules,
inspired by the FLWOR syntax of XQuery expressions; an example is
given in Figure~\ref{fig:imperative}.  This is just syntactic sugar,
entirely eliminated by the compiler front-end.  The ``imperative''
fragment offers several statements: \texttt{skip} (does nothing),
\texttt{for}, \texttt{if}, \texttt{match}, block statements (enclosed
in braces), and variable definitions \texttt{var...in}.  Note that
these are \emph{statements}, and are syntactically different from the
similar \emph{expressions} \texttt{for}, \texttt{var}, \texttt{match},
\texttt{if}.

\begin{figure}[t]
  \small
  \begin{lstlisting}[language=ddlog]
for (region in Region) 
    for (person in Person(.region = region.id))
        var is_audience = is_targeted(person) in
        match (is_audience) {
           true -> TargetAudience(person)
           _    -> skip
        }    
  \end{lstlisting}
  \caption{Imperative-style code defining relation TargetAudience.\label{fig:imperative}}
\end{figure}

\subsection{The standard library}

The DDlog standard library has a growing collection of useful
functions and data structures: some generic functions and data-types,
such as \texttt{min}, string manipulation and conversion to strings,
functions to manipulate vectors, sets, maps (insertion, deletion,
lookup, etc.).

\section{DDlog Implementation}\label{sec-system}

\subsection{The DDlog compilation flow}

Figure~\ref{fig:compiler-flow} shows how DDlog programs are compiled
into native code.  The DDlog compiler generates Rust code, which is
linked with the Differential Dataflow
library~\cite{differential-dataflow} to produce a static or dynamic
library, which exposes an API to invoke the incremental computation at
runtime.  Differential Dataflow is the secret sauce behind DDlog's
excellent incremental performance.  It executes computations
represented as dataflow graphs, consisting of relational operators:
joins, antijoins, unions, aggregations, filters, and flatmaps, defined
over data collections.  Each operator has a highly optimized
incremental implementation, incorporating many generic optimizations.
Operators can be parameterized with arbitrary user-defined logic,
which enables a wide range of programs to be implemented on top of
Differential Dataflow.  Differential Dataflow is an independent
open-source project~\cite{differential-dataflow}, described in a
series of
publications~\cite{timely-dataflow,differential-dataflow-paper} and
online documents~\cite{dd-mdbook,dd-reference}.

\subsection{C and Java interfaces}

The compiled DDlog program exposes a transactional API to add and
delete records from input relations and notify the client about
changes in output relations at runtime.  The API is natively
implemented in Rust, with bindings available for other languages,
currently C and Java.  We use the latter to integrate DDlog in the NSC
control plane.  The Java API to DDlog is packaged as a JAR file, and
supplied together with a dynamically-linked library (DLL) that
comprises the native executable.  The Java API is program-independent,
but the native executable needs to be regenerated every time the DDlog
program is changed. We illustrate the Java API to DDlog with a
skeleton example in Figure~\ref{fig:javaapi}.

\begin{figure}
    \footnotesize
    \begin{lstlisting}[language=Java]
/* Java class that represents a record in the Dependency
 * relation. */
class Genealogy {
  String parent;
  String child;
}

/* Instantiate the DDlog program; add a record to the
 * Dependency relation. */
static void main() {
  DDlogAPI api = new DDlogAPI(2 /* threads */,
               r -> onCommit(r) /* callback */);
  /* Obtain handle to Dependency relation. */
  int table = api.getTable("Genealogy");
  /* Create an instance of Dependency class. */
  Genealogy g = new Genealogy();
  dep.parent = "Mike";
  dep.child  = "John";
  /* Create DDlog insert command. */
  DDlogCommand command = new DDlogCommand(
     DDlogCommand.Kind.Insert,
     table,
     dep);

  /* Execute command as an atomic transaction. */
  int exitcode = api.start();  // start transaction
  DDlogCommand[] commands = new DDlogCommands[1];
  commands[0] = command;
  exitcode = api.applyUpdates(commands);
  /* commit transaction; causes the callback to be 
   * invoked for each new or deleted output record */
  exitcode = api.commit();
  /* Terminate execution of the DDlog program. */
  api.stop();
}

static void onCommit(DDlogCommand command) {
  /* Callback invoked on commit once for each insertion or
   * deletion in an output relation. */
  int outputTable = command.tableId;
  if (command.kind == DDlogCommand.Kind.Insert) {
    Object value = command.getObject();
    // insert into outputTable
  } else {
    // delete from outputTable
  }
}
\end{lstlisting}
\caption{Example program showing the Java API to pass and receive data
  from a DDlog program.\label{fig:javaapi}.}
\end{figure}

\section{Applications}\label{sec:applications}

\subsection{Controller for Virtual Networks}

The most significant DDlog program we have written so far is a
reimplementation of OVN~\cite{ovn} -- a production-grade
virtual network controller used to implement the network substrate
for cloud management systems.

OVN translates a set of network management policies into OpenFlow
rules that have to be installed on the virtual switches in the
network.  The logic is very complicated, comprising tens of input,
output and intermediate relations.

The original program was written in C, and is not fully incremental.
The DDlog implementation has about 6000 lines of code, about the same
size as the original code base, but it is fully incremental.
With the exception of a small number of library functions imported
from C, we were able to implement the entire OVN logic in DDlog.
This would not be feasible with a more traditional dialect of Datalog
that does not support types and expressions, as we relied on these
features extensively in our implementation.

%The original
%program builds an in-memory object model of the database; in that
%implementation joins with a primary key become just pointer
%dereferences.  Moreover, some operations are implemented using
%side-effects and are challenging to do in a declarative model; an
%example is allocation of network addresses, where new addresses are
%allocated using a counter, to prevent old addresses from being reused
%for as long as possible.  The original program generates open-flow
%rules encoded as strings, so this implementation makes extensive use
%of the string interpolation facilities of DDlog.

Preliminary evaluation indicates that the incremental performance is
several orders of magnitude better than the original program for large
networks.  Figure~\ref{fig:ovn_perf} compares the cost of applying a small incremental
change to network reconfiguration for varying network sizes for
C and DDlog versions of the program.

\begin{figure}
    \center
    \includegraphics[width=0.4\columnwidth]{update_plot.pdf}
    \caption{The cost of incremental network updates.\label{fig:ovn_perf}}
\end{figure}

%TODO: how well does it work.

\subsection{Program analysis}

We have evaluated DDlog on a large Datalog program written for the
Souffle Datalog compiler.  The Datalog program performs many
compiler analyses simultaneously, computing their fixed-point.  We
have written a Python program that converts a large subset of the
Souffle Datalog syntax into DDlog (we have manually handled the
missing features).  The input program consists of 580 lines, and the
input dataset consists of 3.5 million tuples.

The non-incremental Souffle Datalog engine used by Doop evaluates the
program on this dataset in 23 seconds, whereas DDlog takes 45 seconds.
After the initial evaluation, DDlog propagates small updates, adding or
removing several input records, in under 10ms.  

A hand-optimized version of the program, provided by Doop developers,
reduces the evaluation time to 1 second by using the join order
optimization.  The DDlog does not currently allow the developer to
control join evaluation order, which prevents us from reproducing the
same optimization.

%TODO: how well does it work?

%\subsection{Firewall management}
%
%We have reimplemented a proprietary network management application in
%DDlog.  The application manages a firewall in a network of switches
%and virtual machines (VMs).  The firewall is driven by a centralized
%policy.  The centralized policy is implemented in a distributed
%fashion by the VMs and network switches, each of which performs
%filtering using local rules.  When the centralized policy changes, the
%local rules have to be updated in all network devices.  The core
%of this program is a graph reachability problem, which is written in a
%few lines of DDlog.  The DDlog program outperformed a hand-optimized
%Java implementation consisting of several thousand lines of code
%by a factor of 3 to 8.

\section{Related work}

There is significant work building Datalog engines for various
purposes: Souffle~\cite{scholz-cc16} (program analysis),
Doop~\cite{Bravenboer-oopsla09} (program analysis),
LogiQL~\cite{Borraz-Sanchez-dlp18} (integer linear programming),
Vadalog~\cite{Bellomarini-vldb18} (existential quantification),
DeALS~\cite{Yang-vldb17} (concurrent execution),
DataFun~\cite{Arntzenius-icfp16} (higher-order functional
programming), NDLog~\cite{loo-cacm09} (declarative networking),
SociaLite~\cite{Seo-vldb13} (large scale graph analysis),
and~\cite{Barany-tods17} (probabilistic programming).  A survey can be
found in~\cite{Maier-book18}.

The only other incremental Datalog engine that we are aware of is a
LogiQL~\cite{Green-pods15}, a commercial product of
LogicBlox~\cite{Aref-sigmod15}.

DDlog is built on top of Differential Dataflow~\cite{dd}; several
interesting declarative query engines were built on top of
DD~\cite{timely-dataflow,differential-dataflow-paper}.

Some of the DDlog features were inspired by .Net
LINQ~\cite{meijer-dpcool03,Meijer-sigmod06}.

There is also significant work on incremental computation systems;
some of these are for much richer computation
models~\cite{acar-05,Carlsson-icfp02,Demetrescu-oopsla11,harkes-ecoop16,Hammer-pldi14},
and some just for relational
models:~\cite{ahmad-vldb12,Szabo-ase016,zhao-icmd17}.

\section{Conclusion and future work}\label{sec-conclusions}

DDlog is a young project.  The language is evolving quickly, driven by
the use cases.  We place paramount importance on language usability;
this is why we have enhanced Datalog with many non-traditional
constructs.  Our goal is to reduce as much as possible the need to
transition between multiple languages when writing large projects.

Our ongoing work on DDlog focuses on continuous improvement of its
performance and memory utilization, as well as use-case-driven evolution
of its syntax, features, and libraries.


\bibliographystyle{abbrv}
\bibliography{top}

\end{document}
